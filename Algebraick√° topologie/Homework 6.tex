\documentclass[12pt,a4paper]{article}
\usepackage[utf8]{inputenc}
\usepackage[T1]{fontenc}
\usepackage[czech]{babel}
\usepackage{a4wide}
\usepackage{amsmath, amsthm, amsfonts, amssymb, graphicx, url, fancyhdr,multicol,enumerate,tikz}
\newcommand{\norm}[1]{\left\lVert#1\right\rVert}

\newcommand{\C}{\mathbb{C}}
\newcommand{\Q}{\mathbb{Q}}
\newcommand{\R}{\mathbb{R}}
\newcommand{\Z}{\mathbb{Z}}
\newcommand{\F}{\mathbb{F}}
%\newcommand{\N}{\mathbb{N}}
\newcommand{\id}{\mathrm{id}}
\DeclareMathOperator{\im}{im}
\DeclareMathOperator{\coker}{coker}
\newcommand{\Hom}{\mathrm{Hom}}
\newcommand{\Max}{\mathrm{Max}}
\newcommand{\disc}{\mathrm{disc}}
\newcommand{\Gal}{\mathrm{Gal}}
\newcommand{\Tr}{\mathrm{Tr}}
\newcommand{\N}{\mathrmACN{N}}
\newcommand{\No}{\mathrm{N}_{\Qbb}^K}
\newcommand{\Ok}{\ensuremath{\mathcal{O}_K}}
\newcommand{\Ol}{\ensuremath{\mathcal{O}_L}}
\newcommand{\Cl}{\ensuremath{\mathcal{C}l}}
\newcommand{\p}{\mathfrak{p}}
\newcommand{\qq}{\mathfrak{q}}
\newcommand{\af}{\mathfrak{a}}
\newcommand{\bb}{\mathfrak{b}}
\newcommand{\rr}{\mathfrak{r}}
\newcommand{\al}{\alpha}
\newcommand{\Mat}{\ensuremath{\text{Mat}(2,\mathbb{Z})}}
\newcommand{\Char}{\mathrm{char }}
\newcommand{\blank}{{-}}
\newcommand{\dd}{\partial}
%\theoremstyle{remark}
\newtheorem*{rem}{Remark}
%\theoremstyle{definition}
\newcounter{exercise}
\newtheorem{ex}[exercise]{Exercise}
%\newenvironment{sol}{\paragraph{Solution:}}{\hfill$\square$}
%\newtheorem*{sol}{Solution}

\begin{document}
\pagestyle{fancy}                      %Pro větší­ možnosti práce se záhlaví­mi a zápatími
\fancyhf{}                             %"vvyčištění záhlaví a zápatí"                                         
%\renewcommand{\headheight}{25 pt}                  %
\addtolength{\topmargin}{-30 pt}                   %
\setlength{\headsep}{10 pt}                      %
\fancyhead[L]{{\emph{M8130 Algebraic topology 2017, homework 6}}}  %
\fancyhead[R]{{\emph{Vladimír Sedláček}}}                 % Nastavení­ pro titulní­ stranu
%\fancyfoot[L]{Školní rok 2016/2017}                %
%\renewcommand{\footrulewidth}{0.8 pt}              %
\renewcommand{\headrulewidth}{1 pt}                %               %
\renewcommand*{\proofname}{Solution}
%\setcounter{exercise}{2}
\ex\mbox{} 
\begin{proof}
Let $X$ be the CW-complex $S^{2017}_1\vee S^{2017}_2\vee S_1^{2018} \cup e_2^{2018}\cup e^{2019}$ together with the constant attaching maps $S^{2017}_1\to e^0$, $S^{2017}_2\to e^0$ and $S^{2018}\to e^0$, an arbitrary attaching map $\dd D_2^{2018}=S^{2017}\to S_2^{2017}$ of degree $6$ and an arbitrary attaching map $\dd D^{2019}=S^{2018}\to S_1^{2018}$ of degree $4$ (we already know such maps do exist). Then the homology groups of $X$ are the same as the homology groups of the chain complex
$$\dots \to 0 \to 0 \to \underbrace{\Z}_{\dim 2019} \xrightarrow{\cdot\begin{pmatrix}
        4&0
    \end{pmatrix}}\underbrace{\Z\oplus \Z}_{\dim 2018}\xrightarrow{\cdot\begin{pmatrix}
        0&0\\
        0&6\\
    \end{pmatrix}}\underbrace{\Z\oplus\Z}_{\dim 2017}\to 0\to 0\to\dots\to\underbrace{0}_{\dim 1}\to \underbrace{\Z}_{\dim 0}\to 0.$$
Therefore (since the kernel of $\cdot\begin{pmatrix}
        0&0\\
        0&6\\
    \end{pmatrix}$ is $\Z\oplus 0$ and $\cdot\begin{pmatrix}
        4&0
    \end{pmatrix}$ is injective) we have
\begin{equation*}
\begin{split}
H_0(X)&=\Z/0=\Z,\\
H_{2017}(X)&=(\Z\oplus\Z)/(0\oplus 6\Z)=\Z\oplus \Z/6,\\
H_{2018}(X)&=(\Z\oplus 0)/(4\Z\oplus 0)=\Z/4
\end{split}
\end{equation*}
and all the other homology groups are trivial, as required.
\end{proof}
\end{document}
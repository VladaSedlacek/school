\documentclass[12pt,a4paper]{article}
\usepackage[utf8]{inputenc}
\usepackage[T1]{fontenc}
\usepackage[czech]{babel}
\usepackage{a4wide}
\usepackage{amsmath, amsthm, amsfonts, amssymb, graphicx, url, fancyhdr,multicol,enumerate,tikz}
\newcommand{\norm}[1]{\left\lVert#1\right\rVert}

\newcommand{\C}{\mathbb{C}}
\newcommand{\Q}{\mathbb{Q}}
\newcommand{\R}{\mathbb{R}}
\newcommand{\Z}{\mathbb{Z}}
\newcommand{\F}{\mathbb{F}}
%\newcommand{\N}{\mathbb{N}}
\newcommand{\id}{\mathrm{id}}
\DeclareMathOperator{\im}{im}
\DeclareMathOperator{\coker}{coker}
\newcommand{\Hom}{\mathrm{Hom}}
\newcommand{\Max}{\mathrm{Max}}
\newcommand{\disc}{\mathrm{disc}}
\newcommand{\Gal}{\mathrm{Gal}}
\newcommand{\Tr}{\mathrm{Tr}}
\newcommand{\N}{\mathrm{N}}
\newcommand{\No}{\mathrm{N}_{\Qbb}^K}
\newcommand{\Ok}{\ensuremath{\mathcal{O}_K}}
\newcommand{\Ol}{\ensuremath{\mathcal{O}_L}}
\newcommand{\Cl}{\ensuremath{\mathcal{C}l}}
\newcommand{\p}{\mathfrak{p}}
\newcommand{\qq}{\mathfrak{q}}
\newcommand{\af}{\mathfrak{a}}
\newcommand{\bb}{\mathfrak{b}}
\newcommand{\rr}{\mathfrak{r}}
\newcommand{\al}{\alpha}
\newcommand{\Mat}{\ensuremath{\text{Mat}(2,\mathbb{Z})}}
\newcommand{\Char}{\mathrm{char }}
\newcommand{\blank}{{-}}
\newcommand{\dd}{\partial}
%\theoremstyle{remark}
\newtheorem*{rem}{Remark}
%\theoremstyle{definition}
\newcounter{exercise}
\newtheorem{ex}[exercise]{Exercise}
%\newenvironment{sol}{\paragraph{Solution:}}{\hfill$\square$}
%\newtheorem*{sol}{Solution}

\begin{document}
\pagestyle{fancy}                      %Pro větší­ možnosti práce se záhlaví­mi a zápatími
\fancyhf{}                             %"vvyčištění záhlaví a zápatí"                                         
%\renewcommand{\headheight}{25 pt}                  %
\addtolength{\topmargin}{-30 pt}                   %
\setlength{\headsep}{10 pt}                      %
\fancyhead[L]{{\emph{M8130 Algebraic topology, homework 2}}}  %
\fancyhead[R]{{\emph{Vladimír Sedláček}}}                 % Nastavení­ pro titulní­ stranu
%\fancyfoot[L]{Školní rok 2016/2017}                %
%\renewcommand{\footrulewidth}{0.8 pt}              %
\renewcommand{\headrulewidth}{1 pt}                %               %
\renewcommand*{\proofname}{Solution}
%\setcounter{exercise}{2}
\ex\mbox{} 
\begin{proof}
For any $b\in B$, we have $$j(b-pj(b))=jb-jp(jb)=0$$ by the assumption, hence $b-pj(b)\in \ker j=\im i$. Since $i$ is injective, this allows us to define $q(b):=i^{-1}(b-pj(b)).$ Then we have $$qi(a)=i^{-1}(i(a)-pji(a))=i^{-1}(i(a))=a$$ (since $ji=0$) for all $a\in A$, which shows that $qi=\id_A$. Also, for any $b\in B$, we have $$(iq+pj)(b)=i(i^{-1}(b-pj(b)))+pj(b)=b,$$
which shows that $iq+pj=\id_B$ and we are done.
\end{proof}

\ex\mbox{} 
\begin{proof}\mbox{}
\begin{enumerate}[(1)]
\item 
Let $a,b,c$ be as in the formula and let $[c']=[c]\in H_n(C_*)$. Then there exists some $d\in C_{n+1}$ such that $c'=c+\dd d$ and since $g$ is surjective, there must also exist some $b'\in B_{n+1}$ such that $g(b')=d$. Then we have
$$g(b+\dd b')=g(b)+g\dd (b')=c+\dd g(b')=c+\dd d=c'$$
and $$f(a)=\dd b=\dd b+\dd \dd b'=\dd (b+\dd b'),$$
which implies $\dd_*[c']=[a]=\dd_*[c]$.
\item 
If $[c]\in \im g_*$ then there is some $b\in \ker Z_n(B)$ such that $g(b)=c$, so that $$\dd [c]=[f^{-1}(\dd b)]=[f^{-1}(0)]=[0],$$ which shows that $\im g_*\subseteq \ker \dd_*$.\\
Conversely, suppose that $\dd_*[c]=0$, i.e. we have $b\in B_n,a\in A_{n-1}$ such that $g(b)=c, f(a)=\dd b$ and $[a]=0$. Then there must exist $a'\in A_n$ such that $a=\dd a'$, which implies $\dd b=f\dd a'=\dd f(a')$, hence $\dd (b-f(a'))=0$ and $b-f(a')\in Z_n(B)$. Also $g(b-f(a'))=g(b)=c$, which shows that $g_*[b-f(a')]=[c]$. Thus $\im g_*\supseteq \ker \dd_*$.
\item Let $a\in Z_n(a)$. Then $f_*[a]=0$ iff $f(a)=\dd b$ for some $b\in B_{n+1}$ iff $[a]=\dd_*(g_*[b])$, which shows that $\ker f_*=\im \dd_*$ (the last equivalence holds because $\dd g(b)=g\dd (b)=gf(a)=0$, hence $g(b)\in Z_{n+1}(C)$).
\end{enumerate} 
\end{proof}

\end{document}
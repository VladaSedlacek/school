\documentclass[12pt,a4paper]{article}
\usepackage[utf8]{inputenc}
\usepackage[T1]{fontenc}
\usepackage[czech]{babel}
\usepackage{a4wide}
\usepackage{amsmath, amsthm, amsfonts, amssymb, graphicx, url, fancyhdr,multicol,enumerate,tikz}
\newcommand{\norm}[1]{\left\lVert#1\right\rVert}

\newcommand{\C}{\mathbb{C}}
\newcommand{\Q}{\mathbb{Q}}
\newcommand{\R}{\mathbb{R}}
\newcommand{\Z}{\mathbb{Z}}
\newcommand{\F}{\mathbb{F}}
%\newcommand{\N}{\mathbb{N}}
\newcommand{\id}{\mathrm{id}}
\newcommand{\im}{\mathrm{im}}
\newcommand{\cok}{\mathrm{coker}}
\newcommand{\Hom}{\mathrm{Hom}}
\newcommand{\Max}{\mathrm{Max}}
\newcommand{\disc}{\mathrm{disc}}
\newcommand{\Gal}{\mathrm{Gal}}
\newcommand{\Tr}{\mathrm{Tr}}
\newcommand{\N}{\mathrm{N}}
\newcommand{\No}{\mathrm{N}_{\Qbb}^K}
\newcommand{\Ok}{\ensuremath{\mathcal{O}_K}}
\newcommand{\Ol}{\ensuremath{\mathcal{O}_L}}
\newcommand{\Cl}{\ensuremath{\mathcal{C}l}}
\newcommand{\p}{\mathfrak{p}}
\newcommand{\qq}{\mathfrak{q}}
\newcommand{\af}{\mathfrak{a}}
\newcommand{\bb}{\mathfrak{b}}
\newcommand{\rr}{\mathfrak{r}}
\newcommand{\al}{\alpha}
\newcommand{\Mat}{\ensuremath{\text{Mat}(2,\mathbb{Z})}}
\newcommand{\Char}{\mathrm{char }}
\newcommand{\blank}{{-}}
%\theoremstyle{remark}
\newtheorem*{rem}{Remark}
%\theoremstyle{definition}
\newcounter{exercise}
\newtheorem{ex}[exercise]{Exercise}
%\newenvironment{sol}{\paragraph{Solution:}}{\hfill$\square$}
%\newtheorem*{sol}{Solution}

\begin{document}
\pagestyle{fancy}                      %Pro větší­ možnosti práce se záhlaví­mi a zápatími
\fancyhf{}                             %"vvyčištění záhlaví a zápatí"                                         
%\renewcommand{\headheight}{25 pt}                  %
\addtolength{\topmargin}{-30 pt}                   %
\setlength{\headsep}{10 pt}                      %
\fancyhead[L]{{\emph{M8130 Algebraic topology, homework 1}}}  %
\fancyhead[R]{{\emph{Vladimír Sedláček}}}                 % Nastavení­ pro titulní­ stranu
%\fancyfoot[L]{Školní rok 2016/2017}                %
%\renewcommand{\footrulewidth}{0.8 pt}              %
\renewcommand{\headrulewidth}{1 pt}                %               %
\renewcommand*{\proofname}{Solution}
\setcounter{exercise}{2}
\ex\mbox{} 
\begin{proof}
Let $u,v$ be any two distinct points of $X$, so that $(u,v)\in (X\times X) \setminus \Delta$. Since $X$ is Hausdorff, there exist open disjoint sets $U,V\subseteq X$ such that $u\in U, v\in V$. Then $(u,v)\in U\times V\subseteq (X\times X) \setminus \Delta$. Since $U\times V$ is open in the product topology (it belongs to its basis) and $(u,v)\in (X\times X) \setminus \Delta$ were chosen arbitrarily, this means that $(X\times X) \setminus \Delta$ is open, hence $\Delta$ is closed.
\end{proof}

\ex\mbox{} 
\begin{proof}
Consider the map $f:X\to X\times X$ given by $f(x)=(x,r(x))$ (which is clearly continuous) and let $\Delta_A:=\{(a,a)\in A\times A\}$. Since $r(a)=a$ for $a\in A$ by the definition of retraction, it follows that $f(x)\in \Delta_A$ if and only if $x\in A$. Also $A$ is Hausdorff (it is a subset of a Hausdorff space $X$), so $\Delta_A$ is closed by the previous exercise, hence $A=f^{-1}(\Delta_A)$ is closed as well.
\end{proof}

\ex\mbox{} 
\begin{proof}
Since $A \hookrightarrow X$ is a cofibration, there exists a retraction $r:X\times I\to X\times\{0\}\cup A\times I$. Using the previous exercise, this implies that $X\times\{0\}\cup A\times I$ is closed in $X\times I$ (because $X\times I$ is Hausdorff, as a product of two Hasudorff spaces). Therefore $X\times I\setminus(X\times\{0\}\cup A\times I)$ is open, and since the canonical projection $p:X\times I\to X$ is open, the image $$p(X\times I\setminus(X\times\{0\}\cup A\times I))=X\setminus A$$
must be open as well. Thus $A$ is closed.
\end{proof}
\end{document}